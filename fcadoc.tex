\documentclass[12pt]{article}
\usepackage{a4,color}
\usepackage{fca}
\usepackage{cxtinput}
%\selectlanguage{english}
\usepackage[utf8]{inputenc}
\usepackage[T1]{fontenc}
\usepackage[english]{babel}
\usepackage{lmodern}
\usepackage{xcolor}
\usepackage[%
   colorlinks=false,
   allbordercolors={0.9 0.9 0.9},
   breaklinks=true,
   ]{hyperref}
\hypersetup{
   pdfauthor = {Bernhard Ganter},
   pdfkeywords = {FCA LaTeX sytle diagrame context},
   pdftitle = {fca.sty – LaTeX macros for Formal Concept Analysis},
   pdfsubject = {LaTeX style file for typesetting formal contexts and
     concept latticees}
}

\RequirePackage{stmaryrd}
\def\fcastyle{\texttt{fca.sty}\xspace}
\title{\fcastyle\\\LaTeX--macros for Formal Concept Analysis\\
{ Version 2.1}}
\author{Bernhard Ganter\\ TU Dresden}
\date{October 1, 2007}
\begin{document}
\maketitle
\begin{abstract}
\FCA is a mathematical field based on the theory of lattices and ordered
sets, with applications in data analysis and knowledge processing. 
To simplify typesetting of FCA-related text, \fcastyle  provides two
environments and some simple macros, just for convenience. \fcastyle offers
nothing that you could not do without. The two environments are
\begin{description}
\item[cxt] for typesetting small formal contexts as cross-tables, and
\item [diagram] for making line diagrams of concept lattices. This environment
  may be of some interest for other purposes as well, since it can also be
  used for ordered sets and graphs.
\end{description}
A recent version of \texttt{fca.sty} should be available from
\begin{center}
  \texttt{www.math.tu-dresden.de/$\sim$ganter}
\end{center}
\end{abstract}
%
%
\tableofcontents

\section{Environment \texttt{cxt}}
What this (very simple) environment does can be guessed from an example:
The text on the left leads to the output on the right.\bigbreak

\noindent\begin{minipage}{.45\textwidth}
\begin{verbatim}
\begin{cxt}%
\cxtName{Formula 1}%
\att{1.}%
\att{2.}%
\atr{disqualified}%
\obj{x..}{Hamilton}
\obj{.x.}{Alonso}
\obj{.xx}{Massa}
\end{cxt}
\end{verbatim}
\end{minipage}
\hfill
\begin{minipage}{.40\textwidth}
\begin{cxt}%
\cxtName{Formula 1}%
\att{1.}%
\att{2.}%
\atr{disqualified}%
\obj{x..}{Hamilton}
\obj{.x.}{Alonso}
\obj{.xx}{Massa}
\end{cxt}
\end{minipage}
\bigbreak

\texttt{cxt} generates a tabular of the appropriate format. The tabular is
defined as soon as the first \verb|\obj| command is given. Spaces in the 
preceding lines are not ignored (in this version). Therefore, each line should
be ended with a \texttt{\%}\ . (To be repaired in later versions).
\medbreak
 
The commands within a \texttt{cxt}--environment are

\begin{description}
\item[]\verb|\cxtName{}|\quad Define the text for the upper left cell of the
  table. Optional. The default is no text.
\item[]\verb|\att{}|\quad Give an attribute name. These names are processed
  in the order in which they are given. Attribute names given after an
  \verb|\obj|  command are ignored.
\item[]\verb|\atr{}|\quad Same as \verb|\att{}|, but with rotated text.
\item[]\verb|\obj{}{}|\quad Give an object's name and its incidence vector,
  consisting of dots and `x'es. The incidences come first, for better
  alignment. The length of each incidence vector must be the number of
  attributes. 

Each  instance of \verb|\obj| is directly translated to a row of the
\verb|tabular|-environment. It is therefore possible to mix  \verb|\obj|
commands with usual  \verb|tabular|-commands.
\end{description}
\texttt{cxt} can handle an arbitrary number of attributes.
\bigbreak

The arrow relations may also be used. Instead of \texttt{x} and \texttt{.},
type \texttt{d} (for ``down''), \texttt{u} (``up''), or \texttt{b} (``both''), 
as in the following example:\bigbreak 

\noindent\begin{minipage}{.45\textwidth}
\begin{verbatim}
\begin{cxt}%
\renewcommand{\cxtArrowStyle}{\footnotesize\color{red}}
\cxtName{Formula 1}%
\att{1.}%
\att{2.}%
\atr{disqualified}%
\obj{xbd}{Hamilton}
\obj{uxb}{Alonso}
\obj{bxx}{Massa}
\end{cxt}
\end{verbatim}
\end{minipage}
\hfill
\begin{minipage}{.40\textwidth}
\strut\par\strut\par
\begin{cxt}%
\renewcommand{\cxtArrowStyle}{\footnotesize\color{red}}
\cxtName{Formula 1}%
\att{1.}%
\att{2.}%
\atr{disqualified}%
\obj{xbd}{Hamilton}
\obj{uxb}{Alonso}
\obj{bxx}{Massa}
\end{cxt}
\end{minipage}
\bigbreak

The default for \verb|\cxtArrowStyle| is \verb|\footnotesize|. In the above
example we have changed it using \verb|\renewcommand| in order to make the
arrows red. The default colour is black. 

\section{Including Burmeister context files}

The package |cxtinput| allows to use context files in Burmeister
format to be included directly in a \LaTeX{} document. It's usage is
as simple as possible. In order to allow for certain customizations it
needs to be issued inside a |cxt| environment.

\noindent\begin{minipage}{.45\textwidth}
\begin{verbatim}
\documentclass{article}
\usepackage{fca,cxtinput}
\begin{document}
\begin{cxt}%
\cxtName{Formula 1}%
\cxtinput{formula1.cxt}%
\end{cxt}
\end{document}
\end{verbatim}
\end{minipage}
\hfill
\begin{minipage}{.40\textwidth}
\strut\par\strut\par
\begin{cxt}%
\cxtName{Formula 1}%
\cxtinput{formula1.cxt}%
\end{cxt}
\end{minipage}
\bigbreak

Unrotated attribute names can be achieved using the following code:

\noindent\begin{minipage}{.45\textwidth}
\begin{verbatim}
\documentclass{article}
\usepackage{fca,cxtinput}
\begin{document}
\begin{cxt}%
\cxtName{Formula 1}%
\cxtinput{formula1.cxt}%
\end{cxt}

\begin{cxt}%
\cxtName{Formula 2}%
\cxtinput{formula1.cxt}%
\end{cxt}
\end{document}
\end{verbatim}
\end{minipage}
\hfill
\begin{minipage}{.40\textwidth}
\strut\par\strut\par
\begin{cxt}%
\cxtName{Formula 1}%
\renewcommand{\atr}{\att}%
\cxtinput{formula1.cxt}%
\end{cxt}

\bigskip
\begin{cxt}
\cxtName{Formula 2}%
\cxtinput{formula1.cxt}%
\end{cxt}
\end{minipage}
\bigbreak

\section{Environment \texttt{diagram}}
The \texttt{diagram} environment helps typesetting diagrams of concept
lattices, but can be used for ordered sets and graphs as well. Again we start
with a small example (for which we have set \verb|\unitlength 1.2mm|):
\bigbreak

\noindent\begin{minipage}{.55\textwidth}
\begin{verbatim}
\begin{diagram}{40}{55}
\Node{1}{20}{10}
\Node{2}{35}{20}
\Node{3}{5}{30}
\Node{4}{35}{40}
\Node{5}{20}{50}
\Edge{1}{2}
\Edge{1}{3}
\Edge{2}{4}
\Edge{3}{5}
\Edge{4}{5}
%\Numbers
\leftAttbox{3}{2}{2}{1.}
\rightAttbox{2}{2}{2}{disqualified}
\rightAttbox{4}{2}{2}{2.}
\leftObjbox{3}{2}{2}{Hamilton}
\rightObjbox{2}{2}{2}{Massa}
\rightObjbox{4}{2}{2}{Alonso}
\end{diagram}
\end{verbatim}
\end{minipage}\hfill
\begin{minipage}{.45\textwidth}
{\unitlength 1.2mm\begin{diagram}{40}{55}
\Node{1}{20}{10}\Node{2}{35}{20}\Node{3}{5}{30}\Node{4}{35}{40}\Node{5}{20}{50}
\Edge{1}{2}\Edge{1}{3}\Edge{2}{4}\Edge{3}{5}\Edge{4}{5}
 %\Numbers
 \leftAttbox{3}{2}{2}{1.} \rightAttbox{2}{2}{2}{disqualified}
 \rightAttbox{4}{2}{2}{2.} \leftObjbox{3}{2}{2}{Hamilton}
 \rightObjbox{2}{2}{2}{Massa} \rightObjbox{4}{2}{2}{Alonso}
\end{diagram}}
\end{minipage}
\bigbreak

Here are the commands of the \texttt{diagram}--environment:
\begin{description}
\item[]\verb|\begin{diagram}{width}{height}|\quad
translates to \par
\verb|\begin{picture}(width,height)(\diagramXoffset,\diagramYoffset)|.


The offsets are zero by default. They can be modified using
\verb|\renewcommand|. Note that the diagram dimensions do not take the lables
into account, these may overlap. Putting an \verb|\fbox| around the above
diagram yields (with \verb|\unitlength .7mm|)
\begin{center}
  \fbox{\unitlength .7mm\begin{diagram}{40}{55}
\Node{1}{20}{10}\Node{2}{35}{20}\Node{3}{5}{30}\Node{4}{35}{40}\Node{5}{20}{50}
\Edge{1}{2}\Edge{1}{3}\Edge{2}{4}\Edge{3}{5}\Edge{4}{5}
 %\Numbers
 \leftAttbox{3}{2}{2}{1.} \rightAttbox{2}{2}{2}{disqualified}
 \rightAttbox{4}{2}{2}{2.} \leftObjbox{3}{2}{2}{Hamilton}
 \rightObjbox{2}{2}{2}{Massa} \rightObjbox{4}{2}{2}{Alonso}
\end{diagram}}\end{center}
\item[]\verb|\Node{nodenumber}{xpos}{ypos}|\quad
Puts a circle at position \texttt{(xpos,ypos)} of the picture. 
These circles are drawn when \verb|\end{diagram}| is invoked. The default
diameter of the circles is 4 (times \verb|\unitlength|). It can be changed
(for all circles) with \verb|\CircleSize{}|. The argument must be an integer.
The node numbers must be different, consecutive between 0 and 51, but need not
necessarily be given in ascending order. 

\item[]\verb|\Numbers|\quad Puts numbers inside circles.  While working on a
  diagram it can be helpful to have a picture with numbered 
nodes. The result of the following command sequence is shown on the right:

\begin{minipage}{.6\textwidth}
\begin{verbatim}
\fbox{\unitlength .7mm
\begin{diagram}{40}{55}
\Node{5}{20}{10}
\Node{6}{35}{20}
\Node{4}{5}{30}
\Node{8}{35}{40}
\Node{7}{20}{50}
\Numbers
\end{diagram}}
\end{verbatim}
\end{minipage} \hfill
\begin{minipage}{.25\textwidth}
\fbox{\unitlength .7mm\begin{diagram}{40}{55}
\Node{5}{20}{10}\Node{6}{35}{20}\Node{4}{5}{30}\Node{8}{35}{40}\Node{7}{20}{50}
\Numbers\end{diagram}}\end{minipage}

We recommend to remove the \verb|\Numbers|--command when the diagram is
ready. In most cases it is not a good idea to put text inside the nodes of a diagram. 

\item[]\verb|\Edge{nodenumber1}{nodenumber2}|\quad
Puts a line between the two nodes with the given numbers. These must have been
declared earlier with a \verb|\Node|--command. For nodes with coordinates
\verb|(u,v)| and \verb|(x,y)| the command translates to 
\begin{center}\verb|\fcadrawline(u,v)(x,y)|.\end{center}
\verb|\fcadrawline(u,v)(x,y)| is a pdf\LaTeX--compatible reimplementation of
the \verb|\drawline| command, the latter provided by the \texttt{eepic}
package. 
 
The \verb|\Edge|--command is executed immediately. It can be mixed with other
\texttt{picture}- and \texttt{eepic}--commands like \verb|\spline|
(see the \texttt{eepic} manual).  

\item[]\verb|\leftAttbox{nodenumber}{xoffset}{yoffset}{text1\\ text2\\ ...}|\quad\par
This is one of six commands 
\begin{center}
  \{\verb|\left|, \verb|\center|, \verb|\right|\}$\times$\{\verb|Attbox|,
  \verb|Objbox|\}.\end{center}  These are used to put
text to diagram nodes. The \verb|Attbox|--commands place the text above the
corresponding node, the \verb|Objbox| below. Similarly, the text can be placed
to the left, be centered, or be placed to the right of the labelled node. All
this can be modified with the \texttt{xoffset, yoffset}--parameters. 

The offsets increase the placement effect. A \verb|\rightObjbox|, which is
placed to the lower right of the corresponding node, will be moved even
further to the lower right if the offsets are positive. Similarly, positive
offsets will push a \verb|\leftAttbox| even more to the upper left, etc.

The text of the label is put in a \verb|\parbox|. It can be broken into
several lines using \verb|\\|. The width of the \verb|\parbox| is
\verb|\LabelBoxWidth|, with a default value of 40mm. This can be changed using
\verb|\renewcommand|.

The label text and the labelled node are connected with a dotted line.
Here is an example:

\begin{minipage}{.6\textwidth}
\begin{verbatim}
{\unitlength .7mm
\begin{diagram}{40}{15}
\Node{0}{20}{10}
\leftAttbox{0}{1}{1}{left\\ 
attribute\\ label}
\rightAttbox{0}{10}{10}{right
attri-\\ bute label}
\rightObjbox{0}{20}{5}{right\\ 
object\\ label}
\centerObjbox{0}{0}{5}{centered\\ 
object label}
\end{diagram}}
\end{verbatim}
\end{minipage}\hfill
\begin{minipage}{.3\textwidth}
{\unitlength .7mm
\begin{diagram}{40}{15}
\Node{0}{20}{10}
\leftAttbox{0}{1}{1}{left\\ attribute\\ label}
\rightAttbox{0}{10}{10}{right attri-\\ bute label}
\rightObjbox{0}{20}{5}{right\\ object\\ label}
\centerObjbox{0}{0}{5}{centered\\ object label}
\end{diagram}}
\end{minipage}

The style of the lables is given by
\begin{description}
\item[]\verb|\ObjectLabelStyle|\hspace*{3.5ex}\quad 
Default: \verb|\small\baselineskip6pt\rm|
\item[]\verb|\AttributeLabelStyle|\quad 
Default: \verb|\small\baselineskip6pt\it|.
\end{description}
These values can be changed with \verb|\renewcommand|.

\item[]\verb|\end{diagram}|\quad This concludes the diagram. The circles
  representing the nodes are drawn and filled with white. Everything inside
  such a circle (except for the numbers caused by the \verb|\Numbers| command)
  is erased.
\end{description}

\subsection{Error messages}
Package error messages for the \texttt{diagram} environment are not yet
implemented. Errors usually are caused by using node numbers that have not
been defined earlier.

\subsection{Fine tuning}
You can change certain layout parameters either permanently (by modifying the
file \texttt{fca.sty}) or temporarily using the following commands:
\begin{trivlist}
\item \verb|\CircleSize{}|,  \hspace*{3.5ex}\qquad Default: 4\qquad
(times \verb|\unitlength|)
\item \verb|\NodeColor{}|,  \qquad\qquad Default: white
\item \verb|\NodeThickness{}|,  \qquad Default: 1.2pt
\item \verb|\EdgeThickness{}|,  \qquad Default: .8pt
\item \verb|\NoDots|,
\item \verb|\renewcommand{\ObjectLabelStyle}{}|,
\item \verb|\renewcommand{\AttributeLabelStyle}{}|, 
\item \verb|\renewcommand{\LabelBoxWidth}{}|.    
\end{trivlist}
Except for the first three, these commands can be focussed to single instances,
using brackets. For example,
\begin{center}
\verb|{\NoDots\centerObjbox{nodenumber}{xoffset}{yoffset}{labeltext}}|
\end{center}
generates a single centered object label without dotted line.

\subsection{pdf\LaTeX \ compatibility}
Version 2 of the \texttt{diagram} environment was designed to be pdf\LaTeX\
compatibile. It no longer uses \texttt{eepic.sty}, which is not supported by
pdf\LaTeX. Instead it uses \texttt{pict2e} and a \verb|\fcadrawline| command,
that replaces \texttt{eepic}'s \verb|\drawline|. 

\subsection{Problems with colour}
Since the diagrams are drawn using the \texttt{picture}--commands and the
\texttt{pict2e} package, we can combine with other packages, for example, with
the \texttt{color} package. This allows us to colour edges and text (but not
individual nodes, see \ref{subsec:colornode}).
However, \texttt{color} has a problem with spacing. Changing colors can cause
unwanted spaces, and these are particularly unpleasant in pictures. Have a
look at the following:\medbreak

\begin{minipage}{.6\textwidth}
\begin{verbatim}
{\unitlength 2mm
\begin{diagram}{20}{20}
\Node{1}{5}{5}  
\Node{2}{15}{15}  
{\color{red}\Edge{1}{2}}
{\color{blue}\Edge{1}{2}}
{\color{red}\Edge{1}{2}}
{\color{blue}\Edge{1}{2}}
\end{diagram}}
\end{verbatim}
\end{minipage}\hfill
\begin{minipage}{.3\textwidth}
{\unitlength 2mm
\begin{diagram}{20}{20}
\Node{1}{5}{5}  
\Node{2}{15}{15}  
{\color{red}\Edge{1}{2}}
{\color{blue}\Edge{1}{2}}
{\color{red}\Edge{1}{2}}
{\color{blue}\Edge{1}{2}}
\end{diagram}}
\end{minipage}
\medbreak 

\noindent This effect disappears when spaces are avoided. Here is a better
version:\medbreak 

\begin{minipage}{.6\textwidth}
\begin{verbatim}
{\unitlength 2mm
\begin{diagram}{20}{20}
\Node{1}{5}{5}  
\Node{2}{15}{15}  
{\color{red}\Edge{1}{2}}%
{\color{blue}\Edge{1}{2}}%
{\color{red}\Edge{1}{2}}%
\color{blue}\Edge{1}{2}}%
\end{diagram}
\end{verbatim}
\end{minipage}\hfill
\begin{minipage}{.3\textwidth}
{\unitlength 2mm
\begin{diagram}{20}{20}
\Node{1}{5}{5}  
\Node{2}{15}{15}  
{\color{red}\Edge{1}{2}}%
{\color{blue}\Edge{1}{2}}%
{\color{red}\Edge{1}{2}}%
{\color{blue}\Edge{1}{2}}%
\end{diagram}}
\end{minipage}
%

\subsection{Colouring nodes}
\label{subsec:colornode}
Nodes are filled with white by default. This can be changed to any other colour
using the command \verb|\NodeColor{colorname}|. This color then applies to
\textbf{all nodes}. \texttt{colorname} must be a
color specification as  used by the \texttt{color} package.
\texttt{``red''}, \texttt{``blue''}, and \texttt{``green''} should usually
work. Other colors may be defined with the \verb|\definecolor| command, see
the documentation of the \texttt{graphics} bundle. For finer colour
nuances use the \texttt{xcolor} package and its documentation.
 
\fcastyle does not support individual node colouring, but there is a trick to
do it nevertheless. Simply include the \texttt{diagram} environment into a 
\texttt{picture} environment and insert the coloured nodes after the diagram
is drawn. How this is done should become clear from the example
below. \fcastyle provides a command \verb|\ColorNode{colorname}| for this.
It overwrites numbers generated by the \verb|\Numbers| command.
\medbreak

\begin{minipage}{.6\textwidth}
\begin{verbatim}
{\definecolor{grey}{gray}{.8}
\unitlength 2mm
\NodeThickness{2.5pt}
\EdgeThickness{2.5pt}
\begin{picture}(20,20)%
\put(0,0){%
\begin{diagram}{20}{20}
\NodeColor{grey}
\Node{1}{5}{5}  
\Node{2}{15}{15}  
\Edge{1}{2}%
\end{diagram}}
\put(5,5){\ColorNode{green}}
\end{picture}}
\end{verbatim}
\end{minipage}\hfill
\begin{minipage}{.3\textwidth}
{\unitlength 2mm\definecolor{grey}{gray}{.8}
\NodeThickness{2.5pt}
\EdgeThickness{2.5pt}
\begin{picture}(20,20)%
\put(0,0){%
\begin{diagram}{20}{20}
\NodeColor{grey}
\Node{1}{5}{5}  
\Node{2}{15}{15}  
\Edge{1}{2}%
\end{diagram}}
\put(5,5){\ColorNode{green}}
\end{picture}}
\end{minipage}
%
\clearpage
\section{Some macros}
For a short description see Figure~\ref{fig:macros}.
\begin{figure}[p]
  \begin{center}
    \begin{tabular}{|c|l|l|}\hline
Result&command& German version\\\hline
$\GMI$&\verb|\GMI|&\\
$\context$&\verb|\context|&\\
$\context[L]$&\verb|\context[L]|&\\
$\CL$ &\verb|\CL| &\verb|\BV| \\
$\BVGMI$  &\verb|\CLGMI| &\verb|\BVGMI| \\
$\BGMI$  & \verb|\CGMI|& \verb|\BGMI|\\
$\extent{}$&\verb|\extent{}|&\\
$\intent{}$&\verb|\intent{}|&\\
$\extents{}$&\verb|\extents{}|&\\
$\intents{}$&\verb|\intents{}|&\\
$\HNI$   &\verb|\HNI| & \\
$\relI$   &\verb|\relI| & \\
$\notI$   &\verb|\notI| & \\
$\bigtimes$   &\verb|\bigtimes| & \\
$\Semi$   &\verb|\Semi| & \\
$\Runterpfeil$   &\verb|\DownArrow|&\verb|\Runterpfeil| \\
$\Hochpfeil$   &\verb|\UpArrow| &\verb|\Hochpfeil| \\
$\Doppelpfeil$   &\verb|\DoubleArrow| &\verb|\Doppelpfeil| \\
%$\IRunterpfeil$   &\verb|\IDownArrow| &\verb|\IRunterpfeil| \\
$\IHochpfeil$   &\verb|\IUpArrow| &\verb|\IHochpfeil| \\
$\DDPfeil$   &\verb|\DDArrow| &\verb|\DDPfeil| \\
$\NDDPfeil$   &\verb|\NDDArrow| &\verb|\NDDPfeil| \\
\FCA   &\verb|\FCA| & \\
\FBA   & &\verb|\FBA| \\
\FnBA   & &\verb|\FnBA| \\\hline
\end{tabular}
\caption{Table of \texttt{fca.sty}--macros.}\label{fig:macros}
  \end{center}
\end{figure}
\begin{figure}[p]
  \begin{center}
      \begin{tabular}{|l|l|l|}\hline
Symbol&command&package required\\\hline
$\vee$&\verb|\vee|&\\
$\wedge$&\verb|\wedge|&\\
$\bigvee$&\verb|\bigvee|&\\
$\bigwedge$&\verb|\bigwedge|&\\
$\sqcup$&\verb|\sqcup|&\\
$\sqcap$&\verb|\sqcap|&\\
$\bigsqcup$&\verb|\bigsqcup|&\\
$\bigsqcap$&\verb|\bigsqcap|&stmaryrd\\\hline
%&\verb||&\\  
\end{tabular}
\caption{Other symbols that are used in \FCA, and the commands that generate
  them.}\label{fig:2}\end{center} 
\end{figure}
%

\begin{description}
\item[]\verb|\GMI|\quad The formal context $\GMI$. 
\item[]\verb|\context|\quad The symbol $\context$, a frequently used name for
  a formal context.
\item[]\verb|\context[S]|\quad Other letters, such as $\context[S]$, may also
  be used. 
\item[]\verb|\CL|\quad The symbol $\CL$ for the concept lattice operator. If
  ${\mathbb K}$ is a formal context, then $\CL({\mathbb K})$ denotes its
  concept lattice.   
\item[]\verb|\BV|\quad same as \verb|\CL|.  
\item[]\verb|\CLGMI|\quad The concept lattice $\CLGMI$ of the formal context
  $(G,M,I)$.   
\item[]\verb|\BVGMI|\quad Same as \verb|\CLGMI|.  
\item[]\verb|\CGMI|\quad The set $\CGMI$ of all formal concepts of the formal
  context   $(G,M,I)$.   
\item[]\verb|\BGMI|\quad Same as \verb|\CGMI|.  
\item[]\verb|\extent|\quad The extent $\extent{\mathfrak{c}}$ of the formal
  concept $\mathfrak{c}:=(A,B)$ is  $A$. 
\item[]\verb|\intent|\quad The intent $\intent{\mathfrak{c}}$ of the formal
  concept $\mathfrak{c}:=(A,B)$ is  $B$. 
\item[]\verb|\extents|\quad The set $\extents{\context}$ of extents of the
  formal context $\context$.  
\item[]\verb|\intents|\quad The set $\intents{\context}$ of intents of the
  formal context $\context$.  
\item[]\verb|\HNI|\quad The subcontext $\HNI$.  
\item[]\verb|\relI|\quad The incidence relation $\relI$.  
\item[]\verb|\notI|\quad The negation $\notI$ of the incidence
  relation.  
\item[]\verb|\bigtimes|\quad The product symbol $\bigtimes$.
\item[]\verb|\DownArrow|\quad The $\Runterpfeil$ of the arrow relations.  
\item[]\verb|\Runterpfeil|\quad Same as \verb|\DownArrow|. 
\item[]\verb|\UpArrow|\quad The $\Hochpfeil$ of the arrow relations.
\item[]\verb|\Hochpfeil|\quad   Same as \verb|\UpArrow|.
\item[]\verb|\DoubleArrow|\quad  The $\Doppelpfeil$ of the arrow relations.
\item[]\verb|\Doppelpfeil|\quad   Same as \verb|\DoubleArrow|.
%\item[]\verb|\IDownArrow|\quad Gives $\IRunterpfeil$, which has the same
%  meaning as $\DownArrow$,  but is drawn in the other direction. This is
%  needed in the   definition of $\DPfeil$.
%\item[]\verb|\IRunterpfeil|\quad  Same as \verb|\IDownArrow|.
\item[]\verb|\IUpArrow|\quad  Gives $\IHochpfeil$, which is has the same
  meaning as $\UpArrow$,  but is drawn in the other direction. This is needed
  in the  definition of $\DPfeil$.
\item[]\verb|\IHochpfeil|\quad  Same as \verb|\IUpArrow|.
\item[]\verb|\DDArrow|\quad Gives $\DDPfeil$, the symbol for the transitive
  closure of the arrow relations.  
\item[]\verb|\DDPfeil|\quad  Same as \verb|\DDArrow|.
\item[]\verb|\NDDArrow|\quad Gives $\NDDPfeil$ the symbol for the negation of
  $\DDPfeil$.   
\item[]\verb|\NDDPfeil|\quad  Same as \verb|\NDDArrow|.
\item[]\verb|\Semi|\quad Gives $\Semi$, the symbol for the semi-product.  
\item[]\verb|\FCA|\quad Prints ``\FCA''.  In most cases, this command does not
  eat the space following it (thanks to \verb|\xspace|).  
\item[]\verb|\FBA, \FnBA|\quad Print ``Formale(n) Begriffsanalyse''. These
  commands also use \verb|\xspace| so that blanks are preserved.
\item[]Some symbols that are provided by \LaTeX\ are listed in
  Figure~\ref{fig:2}. 
\end{description}
%
\par\noindent
Here is a sample text:
\begin{verbatim}
\FCA offers an elegant way to determine the congruence relations 
of a complete lattice: The congruence lattice of a doubly founded 
concept lattice $\CLGMI$ is isomorphic to $\CL(G,M,\NDDArrow)$.
\end{verbatim}
This translates to:
\begin{center}
  \begin{minipage}{.87\textwidth}
  \FCA offers an elegant way to determine the congruence relations of a
  complete lattice: The congruence lattice of a doubly founded concept lattice
  $\CLGMI$ is   isomorphic to $\CL(G,M,\NDDArrow)$.
  \end{minipage}
\end{center}
%
%
\section{To do}
\begin{itemize}
\item Improve the placement of the dotted lines connecting nodes with
  attribute- and object names.
\item Allow half-shaded nodes in diagrams, and make them
  (optionally) automatic for object- and attribute concepts.
\item Improve the code to avoid unwanted blanks.
\end{itemize}




\end{document}



